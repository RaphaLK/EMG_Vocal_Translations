\chapter{Constraints and Standards}

\section{Constraints}
Understanding and addressing the constraints of our project was critical for ensuring its successful and timely completion. We faced multiple constraints which included budgeting, legal and ethical constraints, post-deployment, and time.
\subsection{Budget}
Our project received a total of \$1750 from the School of Engineering. This meant we were limited to purchasing 1 Cython Biosensing board, which limited our project's capabilities to accurately map sEMG signals into IPA symbols. However, we had enough budget to purchase other needed equipment, such as the electrodes, electrode cables, Li-Ion batteries, and an Arduino board.  
\subsection{Legal and Ethical}
Legal and ethical considerations played a significant role in the project's constraints. One of our concerns was with user data privacy. Our project utilized biometric data collected from consenting individuals. This required upmost compliance to agreements by participants and the IRB protocol. Furthermore, ethical questions regarding the storage and use of sensitive biometric data, such as facial muscle signals, were also addressed through anonymization protocols and secure data handling practices.
\subsection{Maintenance and Post-Deployment}
Maintenance costs and post-deployment usability were also considered. The reliance on affordable components (such as the MyoWare sensors) meant that we were able to add in the aspect of ease of replacement to our project. 

\subsection{Time}
Time constraints were experienced during the development, data collection, and testing phase of this project. Due to the time constraints, we had to focus our efforts to establish an MVP, developing a reliable machine learning model as soon as possible with the possible trade-off of user comfort and project scope (such as multilingual support).


\section{Standards}
This project considered the following standards:

\subsection{ISO Standards}
ISO norms and guidelines are vital for the protection of device reliability and usefulness, especially those manufactured to facilitate the disabled. They give assurance of quality and risk management standards widely accepted and appreciated internationally by users and stakeholders. In the scope of this project, we recognize multiple ISO standards such as ISO 14971, the risk management for medical devices. It is important to the EMG Vocal Translator since the standard recognizes and addresses the possible risks of device failure or incorrect speech. In so doing, we want to avoid any harm, boost user reassurance and trust, and provide the optimum safety and ethical practice levels for the technology used.
ISO, "Medical devices—Application of risk management to medical devices," ISO 14971, 3rd ed., 2019.

\subsection{IEEE Standards}
A related factor in implementing the project is the capability of transferring user information either between devices or in a specific use such as in medical, and this comes under multiple IEEE standards. For example, IEEE 11073 which is a health device communication standard gives protocols on communication of medical and personal health devices. It defines communication protocols in data exchange and device commands allowing devices to interconnect and share data with other healthcare systems. This makes interoperability, reliability as well as security of handling sensitive health information assured.
IEEE, "Health informatics—Personal health device communication," IEEE Standard 11073, 2022. [Online]. Available: https://standards.ieee.org
