\chapter{Societal Issues}
If you do not think an issue, such as ethics for example, has any relation to your project, you can say so, but you should justify this. SELECT AT LEAST FIVE OF THESE ISSUES TO ADDRESS IN AT LEAST ONE PARAGRAPH EACH:

\section{Ethical}
Our goal of developing a silent-speech interface requires considerable thought into ethics.
The resulting project targets a subset of people with an inability to speak. This means our project
 will be entrusted with sensitive health information and minimal risk to the user. Therefore, we must
ensure that proper code of ethical conducts are followed. Some examples include physical risk, data 
handling, informed consent for data collection, data bias, and accessibility.

Physical risk includes discomfort, irritation, and potential misuse of our device. Users are advised about
 exposure to silver from the electrodes used to obtain the sEMG signals from facial muscle contraction. They are also
 informed about the possible discomfort from the application of the electrodes unto the skin. The users must be aware
 that application of the electrodes onto facial hair will both reduce the accuracy of our device and may result in 
 damage to the facial hair itself.

The ethical constraints of our data handling and collection is a crucial point of discussion within this project.
 Data must be collected from a diverse set of users to avoid user bias. We must balance the transparency of the procedures 
 of our data collection methods and the anonymized nature of the sensitive data we collect. This is to ensure the safety and privacy
 of our test group and ensure that our device remains accessible in nature.

We address and consider these ethical considerations heavily in our project. We ensure that data collected from participants are used
 solely for testing, and not training. They are informed of the risks of the study beforehand, and we provide as much transparency as
 possible regarding our anonymization protocols, ensuring the participants that no identifiable information is stored.

\section{Social}
Engineering is done within a social context, within a community of other people. Sometimes that community is defined very narrowly, sometimes very broadly. A focus on social issues allows us to consider the impact of our work on society. If we develop this product, or implement this system, what will be the effect? All of our human developments, in engineering and elsewhere, have unanticipated consequences, some good, some bad. We have an obligation to reflect on these consequences as well as we are able.

\section{Political}
Many of our projects are very political in nature, requiring us to take into consideration the will of the general public, usually through elected representatives. Engineers who work on public projects need to understand the political processes that make such work possible. What is the potential impact of your project or this type of project on society?

\section{Economic}
Economic considerations in engineering concern the costs of the various steps in the project. Such costs are usually dependent on the engineering decisions that are made during the design phase. Alternative approaches may offer cost options. We also need to consider the cost of money. How do we pay for the cost of a product development? If we must borrow significant amounts of money how do we account for the cost of the loan in the pricing of the product? What economic considerations arose in your project?

\section{Health and Safety}
We develop our products for the use of the public. Hence we must consider health and safety issues related to our product. How safe does a product have to be? Are there laws that determine this? Are there related ethical issues? What health effects are relevant? What health and safety issues arose in your project?

\section{Manufacturability}
Manufacturability issues are of great importance. Can the product be built? Is there an easier way to build this product than first imagined? What development time issues arise? What are the cost issues? Could your project be manufactured? What problems might arise?

\section{Sustainability} 
Sustainability means two things in engineering; one is a narrow sense, one broad sense. In the former sense sustainability refers to the degree to which a product that is developed can continue to be viable and useful for a reasonable amount of time. A product that fails soon after it is built and cannot be repaired or updated or modified to fit new needs is not a sustainable product. In the broader sense a community or region or a world, perhaps, that uses its resources effectively so that it can sustain its life for a long time is said to be sustainable. We say that such a community has a sustainable economy. Engineering can help develop sustainable economies. What sustainability issues arise?

\section{Environmental Impact} 
All of our products and systems have some environmental impact, in the uses of valuable resources, or in the production of pollution, or in other changes in our surroundings. The engineer is obliged to consider such impacts, and to point them out where they arise, or are a threat. What are the environmental issues related to your product?

\section{Usability} 
Usability refers to what is sometimes called “user-friendliness.” Is the device straightforward, easily learned and easily used by the end user. Is your product usable?

\section{Lifelong learning}
Lifelong learning is a necessary part of all professions. You wouldn’t want to have a doctor who did not know the latest procedures and medications to protect your life. And you wouldn’t want an engineer who didn’t know the analysis tools that had been developed since graduation, or the cost-effective materials that had just come along. We just have to keep up. Learning never stops. Did this project help prepare you for the time when you will have to learn on your own, or did it inspire you to study new material?

\section{Compassion}
One definition for compassion is an awareness of and sympathy for the suffering of another. Compassion means to recognize the suffering of another. But let’s look at a broader definition. Let’s define compassion as “the awareness of and sympathy for the suffering of another, and the desire to relieve that suffering.” What does that have to do with engineering? Simple! One of the things that engineers can choose to do in life is to look for and try to relieve suffering where they find it. Perhaps it means replacing an ancient water supply system that is leading to disease in some tiny village, or designing a communication system to protect seniors with illnesses, or designing prosthetic devices for crippled children. Even if we do not decide to make the relieving of suffering the focus of our life’s work, it is still critically important to our fullness as a human being that we feel compassion for the suffering. It is a part of the education that we hope you acquired at Santa Clara.
