\chapter{Societal Issues}
\section{Ethical}
Our goal of developing a silent speech interface requires considerable thought into ethics.
The resulting project targets a subset of people with an inability to speak. This means our project
 will be entrusted with sensitive health information and minimal risk to the user. Therefore, we must
ensure that proper code of ethical conducts are followed. Some examples include physical risk, data 
handling, informed consent for data collection, data bias, and accessibility.

Physical risk includes discomfort, irritation, and potential misuse of our device. Users are advised about
 exposure to silver from the electrodes used to obtain the sEMG signals from facial muscle contraction. They are also
 informed about the possible discomfort from the application of the electrodes unto the skin. The users must be aware
 that application of the electrodes onto facial hair will both reduce the accuracy of our device and may result in 
 damage to the facial hair itself.

The ethical constraints of our data handling and collection is a crucial point of discussion within this project.
 Data must be collected from a diverse set of users to avoid user bias. We must balance the transparency of the procedures 
 of our data collection methods and the anonymized nature of the sensitive data we collect. This is to ensure the safety and privacy
 of our test group and ensure that our device remains accessible in nature.

We address and consider these ethical considerations heavily in our project. We ensure that data collected from participants are used
 solely for testing, and not training. They are informed of the risks of the study beforehand, and we provide as much transparency as
 possible regarding our anonymization protocols, ensuring the participants that no identifiable information is stored.

\section{Social}
This project has great social impact. Our goal is to improve and provide an additional interface of communication for the vocally impaired.
 Thus enhancing the quality of life and improving social integration of a marginalized subset of people. The device would reduce the isolation
 experienced by the vocally impaired. It allows those individuals to more easily engage in daily activities, such as daily conversation, educational and professional settings. 
 Thus promoting inclusivity in social settings and sheds light on the challenges faced by them, creating a more equitable society.

\section{Health and Safety}
Our device heavily leans towards the definition of a medical device. We are directly utilizing bioelectric signals
 in order to create a silent speech interface. The device must ensure minimal risk to the user, and it must secure any
 identifying health information obtained from the user. Some examples of Health and Safety standards can be found in the previous
 section on ISO/IEEE Standards, namely ISO 14971 and IEEE 11073.

\section{Usability} 
Usability will be a deciding factor for the efficacy of our device for both short-term and long-term usage. Our device design aims to be simple to use and activate,
 with a comfort level in which a user would be able to utilize our device for long periods of time with minimal discomfort and no irritation. While our project may be bounded by constraints such
 as the generalization of musculosignals, we aim to utilize a calibration mechanic in order to individualize our device. The societal impact of a usable silent speech interface would
 improve the quality of life of those with vocal impairments.

\section{Lifelong learning}
Lifelong learning is a key aspect to our device's core design. The device provides an additional method for the vocally impaired community to interact and socialize with others.
 Not only does it allow such individuals to better fit into society, but it also spreads awareness of the challenges vocally impaired individuals face on a daily basis. We hope to
 empower our users and promote independence in their daily lives.

\section{Compassion}
Our project is built upon compassion for those with vocal impairments. By providing an interface that enables the fundamental human
 ability of speech, we empower those with vocal impairments to participate in society and express themselves. Our device recognizes
 the challenges felt by our users. The inability or difficulty to speak can lead to feelings of isolation, or frustration. Through
 our commitment to comfort, usability, personalization, and safety, we aim to address these challenges and spread awareness about
 the struggles of the vocally impaired.